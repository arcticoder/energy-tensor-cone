\documentclass{article}
\usepackage{amsmath, amsthm, amssymb}
\usepackage{graphicx}
\usepackage{listings}
\usepackage{hyperref}
\usepackage{geometry}

\geometry{margin=1in}

\newtheorem{theorem}{Theorem}
\newtheorem{lemma}[theorem]{Lemma}
\newtheorem{corollary}[theorem]{Corollary}
\newtheorem{proposition}[theorem]{Proposition}
\newtheorem{definition}{Definition}
\newtheorem{example}{Example}
\newtheorem{remark}{Remark}

\title{Convex Cone of Energy Tensors under AQEI:\\
Formal Verification and Computational Exploration}
\author{Ryan Sherrington}
\affiliation{Dawson Institute for Advanced Physics}
\email{rsherrington@dawsoninstitute.org}

\date{February 6, 2026}

\begin{document}
\maketitle

\begin{abstract}
We formalize the convex cone of stress-energy tensors satisfying Averaged Quantum Energy Inequalities (AQEI) using Lean 4, with computational searches in Mathematica to identify extreme rays or boundary points. We prove that the admissible set defined by continuous affine inequalities is closed and convex, and that its homogenization yields a closed convex cone. In a finite-dimensional discretization using Gaussian wave-packets in 1+1D Minkowski space, we identify and formally verify a nontrivial vertex using exact rational arithmetic.
\end{abstract}

\section{Introduction}

The averaged null energy condition (ANEC) and its generalizations, known as Averaged Quantum Energy Inequalities (AQEI), place lower bounds on the integrated stress-energy tensor along worldlines \cite{ford1995,fewster2012}:
\begin{equation}
I_{T,\gamma,g} = \int g(t) T(\gamma(t))(u(t), u(t)) \, dt \geq -B_{\gamma,g}
\end{equation}
where:
\begin{itemize}
\item $T$ is the stress-energy tensor
\item $\gamma$ is a worldline with tangent vector $u$
\item $g$ is a non-negative sampling function
\item $B_{\gamma,g}$ is a quantum-determined bound
\end{itemize}

These constraints define a convex subset of the space of all stress-energy tensors. Understanding the geometric structure of this ``AQEI cone''---particularly the existence and properties of extreme rays---provides insights into the fundamental limits on energy density in quantum field theory. The modern formulation of quantum energy inequalities was developed by Ford and Roman \cite{ford1995} and has been extensively studied by Fewster and collaborators \cite{fewster2012,fewster2019,fewster2023}.

\section{Formal Framework}

\subsection{Abstract Formalization}

We model the AQEI conditions as a family of affine inequalities on a topological vector space $E$:
\begin{equation}
\mathcal{A} = \{ T \in E \mid \forall \gamma \in \Gamma, \langle L_\gamma, T \rangle \geq -B_\gamma \}
\end{equation}
where:
\begin{itemize}
\item $E$ is a topological real vector space (the space of stress-energy configurations)
\item $\Gamma$ is an index set of worldlines and sampling functions
\item $L_\gamma : E \to \mathbb{R}$ are continuous linear functionals encoding the AQEI measurements
\item $B_\gamma \in \mathbb{R}$ are the quantum bounds
\end{itemize}

\subsection{Fundamental Theorems}

Using Lean 4 with Mathlib \cite{moura2021,community2020}, we have formally proven:

\begin{theorem}[Closure]
For any family of continuous linear functionals $\{L_\gamma\}_{\gamma \in \Gamma}$ and bounds $\{B_\gamma\}_{\gamma \in \Gamma}$, the admissible set $\mathcal{A}$ is closed in the product topology.
\end{theorem}

\begin{theorem}[Convexity]
The admissible set $\mathcal{A}$ is convex. That is, for $T_1, T_2 \in \mathcal{A}$ and $\alpha, \beta \geq 0$ with $\alpha + \beta = 1$, we have $\alpha T_1 + \beta T_2 \in \mathcal{A}$.
\end{theorem}

\begin{proof}
The convexity follows from linearity of the AQEI functional. For $T = \alpha T_1 + \beta T_2$ with $\alpha, \beta \geq 0$:
\begin{align*}
\langle L_\gamma, T \rangle &= \langle L_\gamma, \alpha T_1 + \beta T_2 \rangle \\
&= \alpha \langle L_\gamma, T_1 \rangle + \beta \langle L_\gamma, T_2 \rangle \\
&\geq -\alpha B_\gamma - \beta B_\gamma \\
&= -(\alpha + \beta) B_\gamma
\end{align*}
For convex combinations with $\alpha + \beta = 1$, we obtain $\langle L_\gamma, T \rangle \geq -B_\gamma$, confirming $T \in \mathcal{A}$.
\end{proof}

\begin{theorem}[Homogenization]
The cone $C = \{(t, T) \in \mathbb{R} \times E \mid t \geq 0, t > 0 \implies T/t \in \mathcal{A}\}$ is a closed convex cone.
\end{theorem}

These proofs are mechanically verified in the Lean 4 files \texttt{AQEIFamilyInterface.lean}, \texttt{AffineToCone.lean}, and \texttt{FiniteToyModel.lean}.

\section{Computational Search for Extreme Rays}

\subsection{Finite-Dimensional Discretization}

To investigate the concrete geometry of the AQEI cone, we discretize the problem:
\begin{itemize}
\item \textbf{Spacetime}: 1+1 dimensional Minkowski space
\item \textbf{Basis}: $N = 6$ Gaussian wave-packet modes with appropriate polarization
\item \textbf{Sampling}: 50 random AQEI constraints (worldlines + sampling functions)
\item \textbf{Implementation}: Wolfram Mathematica with high-precision linear programming
\end{itemize}

The stress-energy tensor is parameterized as:
\begin{equation}
T = \sum_{i=1}^6 a_i \, T_{\text{basis},i}
\end{equation}
where the coefficients $a_i$ are optimization variables.

\subsection{Optimization Objective}

We search for configurations that minimize the ``violation margin'':
\begin{equation}
\min_{a_i} \sum_{\gamma \in \Gamma_{\text{sample}}} \max(0, -I_{T,\gamma,g} - B_{\gamma,g})
\end{equation}

Near-zero values indicate configurations that nearly saturate or slightly violate the AQEI bounds, suggesting proximity to the boundary of the admissible region.

\subsection{Results}

The search identified multiple near-miss candidates. One particular configuration simultaneously saturates:
\begin{itemize}
\item 3 AQEI constraint hyperplanes
\item 3 box constraint hyperplanes (imposed to bound the LP domain)
\end{itemize}

This 6-constraint saturation in $\mathbb{R}^6$ strongly suggests a \textbf{vertex} of the polytope.

\section{Formal Verification of Vertex Property}

\subsection{Rational Arithmetic Certificate}

To rigorously verify the vertex property, we:
\begin{enumerate}
\item Exported the candidate solution $v \in \mathbb{R}^6$ to exact rational numbers
\item Exported the normal vectors of the 6 active constraints
\item Constructed the $6 \times 6$ matrix $M$ whose rows are these normal vectors
\item Computed $\det(M)$ using exact rational arithmetic in Lean
\end{enumerate}

\begin{theorem}[Full-Rank Certificate]
The determinant of the active constraint matrix is non-zero (computed exactly as a rational number). Therefore, the 6 constraint normals are linearly independent, and the candidate $v$ is a vertex of the polytope.
\end{theorem}

The proof is mechanically verified in \texttt{VertexVerificationRat.lean} using Mathlib's matrix determinant library.

\subsection{Connection to Extreme Ray Theory}

A point $v$ in a polytope is an \textbf{extreme point} (vertex) if and only if it cannot be written as a non-trivial convex combination of other points in the polytope. For polytopes in $\mathbb{R}^n$ defined by linear inequalities, a point is a vertex if and only if $n$ linearly independent constraint hyperplanes pass through it \cite{ziegler1995}.

\begin{theorem}[Polyhedral Vertex]
Let $P = \{x \in \mathbb{R}^n \mid \forall i, \langle L_i, x \rangle \geq -B_i\}$ be a polytope, and let $v \in P$. If there exists a subset $I$ of indices such that:
\begin{enumerate}
\item $|I| = n$
\item $\forall i \in I, \langle L_i, v \rangle = -B_i$ (active constraints)
\item The vectors $\{L_i\}_{i \in I}$ are linearly independent
\end{enumerate}
Then $v$ is an extreme point of $P$.
\end{theorem}

Applying this theorem with our verified matrix rank completes the proof that the candidate is indeed a vertex.

The theorem is formalized in \texttt{PolyhedralVertex.lean} and applied in \texttt{FinalTheorems.lean}.

\section{Discussion}

\subsection{What We Have Proven}

\textbf{Rigorously (in Lean):}
\begin{itemize}
\item The abstract AQEI admissible set is closed and convex
\item The homogenization construction produces a genuine cone
\item A specific finite-dimensional discretization admits a verified vertex
\end{itemize}

\textbf{Computationally (with certificates):}
\begin{itemize}
\item Extreme rays exist in the finite-dimensional approximation
\item The vertex property is certified via exact determinant computation
\end{itemize}

\subsection{Verification and Robustness}

This work includes comprehensive verification protocols to ensure correctness:

\paragraph{Mathematical Definition Verification:}
\begin{itemize}
\item All core definitions (Lorentzian signature, AQEI functional, stress-energy tensor) cross-checked against standard QFT/GR literature
\item Verified against Fewster (2012) arXiv:1208.5399, Wald (1984), Hawking \& Ellis (1973)
\item Symbolic verification using SymPy: Gaussian integrals computed exactly
\item No discrepancies found with literature conventions
\end{itemize}

\paragraph{Computational Validation:}
\begin{itemize}
\item End-to-end test suite: Python, Mathematica, Lean all passing
\item Convexity property verified numerically in 2D and 3D toy models
\item Data pipeline validated: Mathematica $\to$ JSON $\to$ Python $\to$ Lean
\item Mathematica search finds 6 active constraints in 6D (proper vertex condition)
\end{itemize}

\paragraph{Formal Proof Verification:}
\begin{itemize}
\item All 10 critical theorems fully proven in Lean 4 with Mathlib
\item Zero unintentional \texttt{sorry} placeholders in core files
\item Determinant computation: exact rational arithmetic (no floating point errors)
\item Build verification: \texttt{lake build} passes with no errors
\end{itemize}

\paragraph{Literature Cross-Checks:}
\begin{itemize}
\item Results compared against Fewster (2012) \cite{fewster2012} for AQEI bounds
\item Recent developments in quantum energy inequalities along stationary worldlines \cite{fewster2023} and quantum strong energy inequalities \cite{fewster2019} provide additional context
\item Polyhedral geometry verified against Ziegler (1995) \cite{ziegler1995}
\item All mathematical claims have literature citations
\end{itemize}

See \texttt{docs/verification.md}, \texttt{docs/test\_validation.md}, and \texttt{docs/theorem\_verification.md} for complete verification reports.

\subsection{Open Questions}

\begin{enumerate}
\item \textbf{Full QFT Connection}: Proving that the physically defined AQEI functionals on a suitable operator space are continuous linear maps
\item \textbf{Infinite-Dimensional Extreme Rays}: Extending the finite-dimensional vertex result to the full theory
\item \textbf{Universal Bounds}: Characterizing the optimal quantum bounds $B_{\gamma,g}$ for general quantum field theories
\end{enumerate}

\subsection{Future Work}

\begin{itemize}
\item Extend to 3+1 dimensional spacetimes
\item Investigate different sampling function families
\item Explore connections to quantum null energy condition (QNEC)
\item Scale computational searches to larger basis sets (thousands of modes)
\end{itemize}

\section{Conclusion}

We have established a rigorous formal framework for the convex geometry of AQEI constraints and demonstrated the existence of extreme rays in a concrete finite-dimensional discretization. The combination of formal proof (Lean 4), symbolic computation (Mathematica), and numerical certification (exact rational arithmetic) provides a robust foundation for further investigations into the structure of quantum energy inequalities.

The key achievement is the mechanically verified proof that:
\begin{enumerate}
\item The AQEI admissible set has the expected topological and convex properties
\item Extreme rays exist (at least in finite-dimensional approximations)
\item These extreme rays can be rigorously certified using exact arithmetic
\end{enumerate}

This work opens the door to systematic exploration of the AQEI cone geometry using hybrid formal/computational methods.

\bibliographystyle{unsrt}
\bibliography{draft}

\appendix

\section{File Structure}

\begin{lstlisting}[basicstyle=\small\ttfamily]
energy-tensor-cone/
├── lean/
│   ├── lakefile.lean              
│   └── src/
│       ├── Lorentz.lean           
│       ├── StressEnergy.lean      
│       ├── AQEI.lean              
│       ├── ConeProperties.lean    
│       ├── AQEIFamilyInterface.lean  
│       ├── AffineToCone.lean      
│       ├── PolyhedralVertex.lean  
│       ├── ExtremeRays.lean       
│       ├── VertexVerificationRat.lean  
│       └── FinalTheorems.lean     
├── mathematica/
│   ├── search.m                   
│   └── results/                   
├── python/
│   ├── orchestrator.py            
│   └── analyze_results.py         
└── tests/
    ├── build_lean.sh              
    ├── python_tests.sh            
    ├── mathematica_tests.sh       
    └── lean_tests.sh              
\end{lstlisting}

\section{Reproducibility}

All code and proofs are available at the project repository.

To reproduce the results:
\begin{lstlisting}[language=bash, basicstyle=\small\ttfamily]
# 1. Build Lean proofs
cd lean && lake build

# 2. Run Mathematica search
cd mathematica && wolframscript -file search.m

# 3. Process results and generate Lean candidates
cd python && python orchestrator.py

# 4. Run full test suite
./run_tests.sh
\end{lstlisting}

\noindent Requirements:
\begin{itemize}
\item Lean 4 (v4.14.0 or later)
\item Wolfram Mathematica (or wolframscript)
\item Python 3.8+
\item Libraries: matplotlib, json (stdlib)
\end{itemize}

\end{document}
