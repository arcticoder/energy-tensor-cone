\section{Introduction}

The averaged null energy condition (ANEC) and its generalizations, known as Averaged Quantum Energy Inequalities (AQEI), place lower bounds on the integrated stress-energy tensor along worldlines \cite{ford1995,fewster2012}:
\begin{equation}
I_{T,\gamma,g} = \int g(t) T(\gamma(t))(u(t), u(t)) \, dt \geq -B_{\gamma,g}
\end{equation}
where:
\begin{itemize}
\item $T$ is the stress-energy tensor
\item $\gamma$ is a worldline with tangent vector $u$
\item $g$ is a non-negative sampling function
\item $B_{\gamma,g}$ is a quantum-determined bound
\end{itemize}

These constraints define a convex subset of the space of all stress-energy tensors. Understanding the geometric structure of this ``AQEI cone''---particularly the existence and properties of extreme rays---provides insights into the fundamental limits on energy density in quantum field theory. The modern formulation of quantum energy inequalities originates with Ford's investigation of quantum coherence effects \cite{ford1978} and was further developed by Ford and Roman \cite{ford1995}, and has been extensively studied by Fewster and collaborators \cite{fewster2012,fewster2019,fewster2023}.

Worldline quantum inequalities provide explicit and general lower bounds for suitably smeared energy densities along timelike curves; see, e.g., Fewster's general framework and ``difference inequality'' formulations \cite{fewster2000}. Variants and refinements include bounds for non-minimally coupled scalar fields \cite{fewster2008}, as well as curved-spacetime stability results demonstrating that flat-spacetime inequalities persist (with controlled corrections) in spacetimes with small curvature \cite{kontou2015}. These results have direct implications for the feasibility of exotic spacetime geometries: for instance, quantum inequalities constrain traversable wormhole geometries \cite{ford1996}, and recent work continues to sharpen such restrictions \cite{kontou2024}.

Alongside analytic developments, there is growing interest in computational and numerical exploration of quantum energy inequalities beyond the simplest free-field settings. Recent numerical investigations in integrable models at the two-particle level \cite{mandrysch2024} and in models with multiple particle species and bound states \cite{bostelmann2024} underscore that QEI/AQEI phenomena can be probed quantitatively in controlled nontrivial theories.

More broadly, inequality constraints and “no-go” tradeoffs also appear in neighboring areas of quantum theory (e.g., in quantum work and fluctuation constraints \cite{hovhannisyan2024}), reinforcing the value of combining computational exploration with mathematically checkable certificates.

Our contribution focuses on the convex-geometric viewpoint: AQEI constraints define an intersection of affine half-spaces, hence a closed convex admissible set, whose homogenization yields a closed convex cone. This perspective draws on standard convex-analytic foundations \cite{rockafellar1970} and polyhedral geometry \cite{ziegler1995}, and supports a hybrid workflow in which (i) computational search proposes boundary candidates, and (ii) formal proof certifies geometric properties of the resulting finite-dimensional models.

\section{Methodology}

Our end-to-end workflow has three components: a finite-dimensional computational model, an auditable artifact pipeline, and a formal certification layer.

\paragraph{Finite-dimensional model.}
We work in 1+1 dimensional Minkowski space, and represent a stress-energy configuration by a coefficient vector $a\in\mathbb{R}^N$ in a fixed Gaussian wave-packet basis.
Given a timelike worldline $\gamma(t)=(t,x_0+vt)$ and a nonnegative sampling function $g$, each sampled AQEI constraint takes the form
\begin{equation}
\langle L_{\gamma,g}, a\rangle \geq -B_{\gamma,g},
\end{equation}
where $L_{\gamma,g}\in\mathbb{R}^N$ is computed by numerical quadrature along $\gamma$.

\paragraph{Sampling and proxy bounds.}
In the current computational search, we use Gaussian sampling functions
$g(t)=\exp(-(t-t_0)^2/(2\tau^2))$ with parameters $(t_0,\tau)$ drawn uniformly from fixed ranges.
For transparency, we also use a simple proxy bound functional
\begin{equation}
B_{\gamma,g}=B_{\mathrm{model}}(g):=0.1\,\|g\|_{L^2([-d,d])},
\end{equation}
as implemented in the Mathematica generator (with integration domain half-width $d$).
Analytic QEIs often yield bounds that can be written in Fourier space; in Fewster's general worldline framework one obtains a state-independent inequality of the form
\begin{equation}
\int d\tau\,|g(\tau)|^2\,\langle :T:\rangle_\omega(\tau,\tau)
\ge
-\int_0^\infty \frac{d\alpha}{\pi}\,\widehat{(g\otimes g)\,\langle T\rangle_{\omega_0}}(-\alpha,\alpha),
\end{equation}
for suitable reference state $\omega_0$ and sampling $g$ \cite{fewster2000}. In contrast, our computational search uses the proxy bound $B_{\mathrm{model}}$ above; Figure~\ref{fig:bound-comparison} compares $B_{\mathrm{model}}$ to a representative analytic scaling for Gaussian sampling.

\begin{figure}[t]
\centering
\includegraphics[width=0.9\linewidth]{figures/bound_comparison.png}
\caption{Bound comparison for Gaussian sampling as a function of width $\tau$. The computational search in this repository uses the proxy bound $B_{\mathrm{model}}(g)=0.1\,\|g\|_{L^2([-d,d])}$ (solid). For context, analytic worldline QEIs typically produce derivative-based bounds with inverse-$\tau$ scaling (dashed; representative Fewster-style benchmark \cite{fewster2000}).}
\label{fig:bound-comparison}
\end{figure}

\paragraph{Artifact pipeline and certification.}
The Mathematica stage exports its sampled constraints and candidate solutions to JSON; Python scripts analyze and visualize these outputs and generate exact (rational) data for Lean. The Lean layer proves closure/convexity results abstractly and certifies that the selected candidate is a vertex of the resulting finite polytope.

\section{Formal Framework}

\subsection{Abstract Formalization}

We model the AQEI conditions as a family of affine inequalities on a topological vector space $E$:
\begin{equation}
\mathcal{A} = \{ T \in E \mid \forall \gamma \in \Gamma, \langle L_\gamma, T \rangle \geq -B_\gamma \}
\end{equation}
where:
\begin{itemize}
\item $E$ is a topological real vector space (the space of stress-energy configurations)
\item $\Gamma$ is an index set of worldlines and sampling functions
\item $L_\gamma : E \to \mathbb{R}$ are continuous linear functionals encoding the AQEI measurements
\item $B_\gamma \in \mathbb{R}$ are the quantum bounds
\end{itemize}

\subsection{Fundamental Theorems}

Using Lean 4 with Mathlib \cite{moura2021,community2020}, we have formally proven:

\begin{theorem}[Closure]
For any family of continuous linear functionals $\{L_\gamma\}_{\gamma \in \Gamma}$ and bounds $\{B_\gamma\}_{\gamma \in \Gamma}$, the admissible set $\mathcal{A}$ is closed in the product topology.
\end{theorem}

\begin{theorem}[Convexity]
The admissible set $\mathcal{A}$ is convex. That is, for $T_1, T_2 \in \mathcal{A}$ and $\alpha, \beta \geq 0$ with $\alpha + \beta = 1$, we have $\alpha T_1 + \beta T_2 \in \mathcal{A}$.
\end{theorem}

\begin{proof}
The convexity follows from linearity of the AQEI functional. For $T = \alpha T_1 + \beta T_2$ with $\alpha, \beta \geq 0$:
\begin{align*}
\langle L_\gamma, T \rangle &= \langle L_\gamma, \alpha T_1 + \beta T_2 \rangle \\
&= \alpha \langle L_\gamma, T_1 \rangle + \beta \langle L_\gamma, T_2 \rangle \\
&\geq -\alpha B_\gamma - \beta B_\gamma \\
&= -(\alpha + \beta) B_\gamma
\end{align*}
For convex combinations with $\alpha + \beta = 1$, we obtain $\langle L_\gamma, T \rangle \geq -B_\gamma$, confirming $T \in \mathcal{A}$.
\end{proof}

\begin{theorem}[Homogenization]
The cone $C = \{(t, T) \in \mathbb{R} \times E \mid t \geq 0, t > 0 \implies T/t \in \mathcal{A}\}$ is a closed convex cone.
\end{theorem}

These proofs are mechanically verified in the Lean 4 files \texttt{AQEIFamilyInterface.lean}, \texttt{AffineToCone.lean}, and \texttt{FiniteToyModel.lean}.

\section{Computational Search for Extreme Rays}

\subsection{Finite-Dimensional Discretization}

To investigate the concrete geometry of the AQEI cone, we discretize the problem:
\begin{itemize}
\item \textbf{Spacetime}: 1+1 dimensional Minkowski space
\item \textbf{Basis}: $N = 6$ Gaussian wave-packet modes with appropriate polarization
\item \textbf{Sampling}: 50 random AQEI constraints (worldlines + sampling functions)
\item \textbf{Implementation}: Wolfram Mathematica with high-precision linear programming
\end{itemize}

The stress-energy tensor is parameterized as:
\begin{equation}
T = \sum_{i=1}^6 a_i \, T_{\text{basis},i}
\end{equation}
where the coefficients $a_i$ are optimization variables.

\subsection{Computational Methodology and Outputs}

The computational stage is designed to be auditable and to produce artifacts that can be independently re-checked.
At a high level, the workflow is:
\begin{enumerate}
\item Sample AQEI constraints (worldlines + sampling parameters) and assemble them into linear inequalities.
\item Solve linear programs over the coefficient vector $a \in \mathbb{R}^6$ to search for near-boundary points.
\item Export candidate points and active constraints to JSON for downstream analysis.
\item Convert the selected candidate and its active constraints into exact rational data and certify the vertex property in Lean.
\end{enumerate}

Concretely, the repository includes representative JSON outputs under \texttt{mathematica/results/} (e.g., \texttt{summary.json} from \texttt{search.m}/\texttt{orchestrator.py} runs, containing aggregate violation and near-miss statistics; \texttt{near\_misses.json}; \texttt{top\_near\_misses.json}; \texttt{violations.json}; and the certified \texttt{vertex.json}), together with the Python and Lean scripts that consume them.

\subsection{Optimization Objective}

We search for configurations that minimize the ``violation margin'':
\begin{equation}
\min_{a_i} \sum_{\gamma \in \Gamma_{\text{sample}}} \max(0, -I_{T,\gamma,g} - B_{\gamma,g})
\end{equation}

Near-zero values indicate configurations that nearly saturate or slightly violate the AQEI bounds, suggesting proximity to the boundary of the admissible region.

\subsection{Results}

The search identified multiple near-miss candidates. One particular configuration simultaneously saturates:
\begin{itemize}
\item 3 AQEI constraint hyperplanes
\item 3 box constraint hyperplanes (imposed to bound the LP domain)
\end{itemize}

This 6-constraint saturation in $\mathbb{R}^6$ strongly suggests a \textbf{vertex} of the polytope.

\begin{figure}[t]
\centering
\includegraphics[width=0.9\linewidth]{figures/vertex_coefficients.png}
\caption{Coefficients $a_i$ of the computationally identified candidate that is later certified (in Lean) to be a vertex of the finite-dimensional feasibility polytope. Three coefficients saturate imposed box constraints (here $a_2\approx a_3\approx a_6\approx 100$), reflecting the role of auxiliary bounds used to keep the linear program bounded and numerically well-conditioned, while the remaining coefficients stay $O(1)$. This binary-like activation pattern is consistent with a polyhedral vertex determined by a small active set of constraints.}
\label{fig:vertex-coefficients}
\end{figure}

\section{Formal Verification of Vertex Property}

\subsection{Rational Arithmetic Certificate}

To rigorously verify the vertex property, we:
\begin{enumerate}
\item Exported the candidate solution $v \in \mathbb{R}^6$ to exact rational numbers
\item Exported the normal vectors of the 6 active constraints
\item Constructed the $6 \times 6$ matrix $M$ whose rows are these normal vectors
\item Computed $\det(M)$ using exact rational arithmetic in Lean
\end{enumerate}

\begin{theorem}[Full-Rank Certificate]
The determinant of the active constraint matrix is non-zero (computed exactly as a rational number). Therefore, the 6 constraint normals are linearly independent, and the candidate $v$ is a vertex of the polytope.
\end{theorem}

The proof is mechanically verified in \texttt{VertexVerificationRat.lean} using Mathlib's matrix determinant library.

\subsection{Connection to Extreme Ray Theory}

A point $v$ in a polytope is an \textbf{extreme point} (vertex) if and only if it cannot be written as a non-trivial convex combination of other points in the polytope. For polytopes in $\mathbb{R}^n$ defined by linear inequalities, a point is a vertex if and only if $n$ linearly independent constraint hyperplanes pass through it \cite{ziegler1995}.

\begin{theorem}[Polyhedral Vertex]
Let $P = \{x \in \mathbb{R}^n \mid \forall i, \langle L_i, x \rangle \geq -B_i\}$ be a polytope, and let $v \in P$. If there exists a subset $I$ of indices such that:
\begin{enumerate}
\item $|I| = n$
\item $\forall i \in I, \langle L_i, v \rangle = -B_i$ (active constraints)
\item The vectors $\{L_i\}_{i \in I}$ are linearly independent
\end{enumerate}
Then $v$ is an extreme point of $P$.
\end{theorem}

Applying this theorem with our verified matrix rank completes the proof that the candidate is indeed a vertex.

The theorem is formalized in \texttt{PolyhedralVertex.lean} and applied in \texttt{FinalTheorems.lean}.

\section{Discussion}

\subsection{What We Have Proven}

\textbf{Rigorously (in Lean):}
\begin{itemize}
\item The abstract AQEI admissible set is closed and convex
\item The homogenization construction produces a genuine cone
\item A specific finite-dimensional discretization admits a verified vertex
\end{itemize}

\textbf{Computationally (with certificates):}
\begin{itemize}
\item Extreme rays exist in the finite-dimensional approximation
\item The vertex property is certified via exact determinant computation
\end{itemize}

\subsection{Verification and Robustness}

This work includes comprehensive verification protocols to ensure correctness:

\paragraph{Mathematical Definition Verification:}
\begin{itemize}
\item All core definitions (Lorentzian signature, AQEI functional, stress-energy tensor) cross-checked against standard QFT/GR literature
\item Verified against Fewster \cite{fewster2012}, Wald \cite{wald1994}, Hawking \& Ellis \cite{hawking1973}
\item Symbolic verification using SymPy: Gaussian integrals computed exactly
\item No discrepancies found with literature conventions
\end{itemize}

\paragraph{Computational Validation:}
\begin{itemize}
\item End-to-end test suite: Python, Mathematica, Lean all passing
\item Test harness: \texttt{./run\_tests.sh} orchestrates the end-to-end pipeline (including a fast Mathematica run); \texttt{tests/python\_tests.sh} additionally smoke-tests code generation on synthetic JSON
\item Bound checks: \texttt{tests/python\_tests.sh} validates the exported vertex certificate (constraint saturation) and sanity-checks both the proxy bound $B_{\text{model}}$ and an analytic scaling benchmark for Gaussian sampling
\item Convexity property verified numerically in 2D and 3D toy models
\item Data pipeline validated: Mathematica $\to$ JSON $\to$ Python $\to$ Lean
\item Mathematica search finds 6 active constraints in 6D (proper vertex condition)
\end{itemize}

\paragraph{Formal Proof Verification:}
\begin{itemize}
\item All 10 critical theorems fully proven in Lean 4 with Mathlib
\item Zero unintentional \texttt{sorry} placeholders in core files
\item Determinant computation: exact rational arithmetic (no floating point errors)
\item Build verification: \texttt{lake build} passes with no errors
\end{itemize}

\paragraph{Literature Cross-Checks:}
\begin{itemize}
\item Results compared against Fewster \cite{fewster2012} for AQEI bounds
\item Recent developments in quantum energy inequalities along stationary worldlines \cite{fewster2023} and quantum strong energy inequalities \cite{fewster2019} provide additional context
\item Polyhedral geometry verified against Ziegler \cite{ziegler1995}
\item All mathematical claims have literature citations
\end{itemize}

See \texttt{docs/verification.md}, \texttt{docs/test\_validation.md}, and \texttt{docs/theorem\_verification.md} for complete verification reports.

\subsection{Verification and Limitations}

While our results provide a rigorous foundation for understanding AQEI cone geometry, several limitations should be acknowledged:

\paragraph{Dimensional Restriction:}
The computational search is performed in 1+1 dimensional Minkowski space. While this simplified setting allows for tractable numerics and clear geometric intuition, the extension to physically realistic 3+1 dimensions remains an open problem. The number of degrees of freedom and constraint complexity scale significantly in higher dimensions.

\paragraph{Finite-Dimensional Approximation:}
We work with a finite Gaussian basis ($N=6$ modes). While the vertex property is rigorously verified in this discretization, the connection to the full infinite-dimensional QFT remains to be established. The finite-dimensional extreme rays identified here may or may not correspond to extreme rays of the full theory.

\paragraph{AQEI Bounds:}
The quantum bounds $B_{\gamma,g}$ used in our computational search are approximate. A full QFT calculation would require detailed analysis of two-point functions and mode expansions, which is beyond the scope of this initial geometric exploration.

In our 1+1D proof-of-concept implementation (Appendix~\ref{sec:keyfiles}), the randomized constraint generator uses a Gaussian sampling family
\begin{equation}
g(t;t_0,\tau)=\exp\bigl(-(t-t_0)^2/(2\tau^2)\bigr)
\end{equation}
and a simple proxy bound
\begin{equation}
B_{\text{model}}(g)=\kappa\,\lVert g \rVert_{L^2},\qquad \kappa=0.1,
\end{equation}
so that $L_{\gamma,g}(a)\ge -B_{\text{model}}(g)$ defines an affine half-space. For an untruncated Gaussian, $\lVert g\rVert_{L^2}^2=\int_{-\infty}^{\infty} e^{-(t-t_0)^2/\tau^2}\,dt=\tau\sqrt{\pi}$, hence $B_{\text{model}}(g)\propto \sqrt{\tau}$.
In the code we sample $\tau\in[0.2,0.8]$ (with a finite integration window), giving a consistent, parameter-dependent family of lower bounds that is sufficient to test the convex-geometric and certification pipeline.

\paragraph{Comparison with Analytic Results:}
Our computational findings are consistent with the general expectation that AQEI/QEI constraints define admissible regions with non-trivial boundary structure. Analytically, worldline QEIs provide bounds of the schematic form
\begin{equation}
\inf_\omega \int d\tau\, (g(\tau))^2\,\rho_\omega(\tau) \ge -\frac{1}{\pi}\int_0^\infty du\,|\widehat{g}(u)|^2\,Q(u)
\end{equation}
for appropriate states $\omega$, sampling functions $g$, and a model-dependent weight $Q$; see \cite{fewster2000} for a general formulation and discussion of difference inequalities. In curved spacetimes, results such as \cite{kontou2015} support the use of flat-spacetime inequalities as a benchmark in regimes of small curvature.

While $B_{\text{model}}(g)$ is not a substitute for a field-theoretic $Q(u)$, it does preserve the key structural feature emphasized in the analytic literature: the bound depends on the sampling profile and its scale, and yields a family of affine constraints indexed by worldline/sampling parameters. Concretely, for the vertex found by the linear program, the three active AQEI constraints are saturated to numerical precision:

\begin{table}[t]
\centering
\begin{tabular}{rrrrr}
\hline
Constraint & $\tau$ & $B_{\gamma,g}$ & $L\!\cdot\!a$ & $L\!\cdot\!a + B_{\gamma,g}$ \\
\hline
23 & 0.64595 & 0.10700 & -0.10700 & $-2.76\times 10^{-15}$ \\
27 & 0.64467 & 0.10689 & -0.10689 & $-6.70\times 10^{-15}$ \\
50 & 0.68728 & 0.11037 & -0.11037 & $\phantom{-}3.25\times 10^{-15}$ \\
\hline
\end{tabular}
\caption{Active AQEI constraints at the computed vertex. Values are computed from the exported certificate data in \texttt{mathematica/results/vertex.json}. The slack $L\!\cdot\!a + B_{\gamma,g}$ is (up to floating-point roundoff) zero, indicating saturation of three AQEI half-spaces. Together with three active box constraints (used to bound the linear program domain), this yields a 6-hyperplane intersection in $\mathbb{R}^6$, consistent with the vertex certificate formalized in Lean.}
\label{tab:active-aqei-constraints}
\end{table}

In this paper we do not attempt a full analytic-to-numeric error budget (e.g., computing the exact $Q(u)$ corresponding to our discretized Gaussian-mode model). Rather, we use the analytic literature to justify the constraint form and to motivate the computational search as a boundary-finding tool. Establishing tighter quantitative comparisons in specific field models (and connecting them to geometric features such as extreme rays) is a natural next step.

Despite these limitations, the hybrid formal/computational approach demonstrates the feasibility of rigorous verification for geometric properties of quantum energy constraints, opening avenues for future work in higher dimensions and full QFT settings.

\subsection{Open Questions}

\begin{enumerate}
\item \textbf{Full QFT Connection}: Proving that the physically defined AQEI functionals on a suitable operator space are continuous linear maps
\item \textbf{Infinite-Dimensional Extreme Rays}: Extending the finite-dimensional vertex result to the full theory
\item \textbf{Universal Bounds}: Characterizing the optimal quantum bounds $B_{\gamma,g}$ for general quantum field theories
\end{enumerate}

\subsection{Future Work}

\begin{itemize}
\item Extend to 3+1 dimensional spacetimes
\item Investigate different sampling function families
\item Explore connections to quantum null energy condition (QNEC)
\item Scale computational searches to larger basis sets (thousands of modes)
\end{itemize}

\section{Conclusion}

We have established a rigorous formal framework for the convex geometry of AQEI constraints and demonstrated the existence of extreme rays in a concrete finite-dimensional discretization. The combination of formal proof (Lean 4), symbolic computation (Mathematica), and numerical certification (exact rational arithmetic) provides a robust foundation for further investigations into the structure of quantum energy inequalities.

The key achievement is the mechanically verified proof that:
\begin{enumerate}
\item The AQEI admissible set has the expected topological and convex properties
\item Extreme rays exist (at least in finite-dimensional approximations)
\item These extreme rays can be rigorously certified using exact arithmetic
\end{enumerate}

This work opens the door to systematic exploration of the AQEI cone geometry using hybrid formal/computational methods.

\section*{Data Availability}

All code, formal proofs, computational data, and supplementary materials for this work are publicly available:
\begin{itemize}
\item \textbf{GitHub repository}: \url{https://github.com/DawsonInstitute/energy-tensor-cone} — complete source code, Lean proofs, Mathematica search scripts, and test suites
\item \textbf{Zenodo archive}: DOI 10.5281/zenodo.18522457 (restricted during review / revision) — persistent versioned snapshot with manuscript, supplements, and reproducibility documentation
\item \textbf{Lean formalization}: All 10 core theorems mechanically verified with zero unintentional \texttt{sorry} placeholders
\item \textbf{Computational results}: Raw JSON outputs from Mathematica search, Python analysis scripts, and generated Lean candidate files included in repository
\end{itemize}

The complete pipeline is reproducible via \texttt{./run\_tests.sh}. See Appendix~\ref{sec:reproducibility} for detailed instructions.

\bibliographystyle{\AQEIBibStyle}
\bibliography{aqei-cone-formalization}

\appendix

\section{Key Files}
\label{sec:keyfiles}

The computational and formal verification pipeline consists of:
\begin{itemize}
\item \textbf{Lean proofs}: \texttt{lean/src/FinalTheorems.lean} (main vertex theorem), \texttt{AffineToCone.lean} (homogenization), \texttt{PolyhedralVertex.lean} (extreme point characterization), \texttt{VertexVerificationRat.lean} (exact rational determinant)
\item \textbf{Computational search}: \texttt{mathematica/search.m} (randomized Gaussian-basis LP solver)
\item \textbf{Data processing}: \texttt{python/orchestrator.py}, \texttt{python/analyze\_results.py} (JSON parsing and Lean code generation)
\item \textbf{Test harness}: \texttt{run\_tests.sh} (end-to-end validation)
\end{itemize}

Complete file structure and detailed documentation available at \url{https://github.com/DawsonInstitute/energy-tensor-cone}.

\section{Reproducibility}
\label{sec:reproducibility}

All code and proofs are available at the project repository.

To reproduce the results:
\begin{lstlisting}[language=bash, basicstyle=\small\ttfamily]
# 1. Build Lean proofs
cd lean && lake build

# 2. Run Mathematica search
cd mathematica && wolframscript -file search.m

# 3. Process results and generate Lean candidates
cd python && python orchestrator.py

# 4. Run full test suite
./run_tests.sh
\end{lstlisting}

\noindent Requirements:
\begin{itemize}
\item Lean 4 (v4.14.0 or later)
\item Wolfram Mathematica (or wolframscript)
\item Python 3.8+
\item Libraries: matplotlib, json (stdlib)
\end{itemize}
